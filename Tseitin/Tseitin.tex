
\documentclass{article}
%% ODER: format ==         = "\mathrel{==}"
%% ODER: format /=         = "\neq "
%
%
\makeatletter
\@ifundefined{lhs2tex.lhs2tex.sty.read}%
  {\@namedef{lhs2tex.lhs2tex.sty.read}{}%
   \newcommand\SkipToFmtEnd{}%
   \newcommand\EndFmtInput{}%
   \long\def\SkipToFmtEnd#1\EndFmtInput{}%
  }\SkipToFmtEnd

\newcommand\ReadOnlyOnce[1]{\@ifundefined{#1}{\@namedef{#1}{}}\SkipToFmtEnd}
\usepackage{amstext}
\usepackage{amssymb}
\usepackage{stmaryrd}
\DeclareFontFamily{OT1}{cmtex}{}
\DeclareFontShape{OT1}{cmtex}{m}{n}
  {<5><6><7><8>cmtex8
   <9>cmtex9
   <10><10.95><12><14.4><17.28><20.74><24.88>cmtex10}{}
\DeclareFontShape{OT1}{cmtex}{m}{it}
  {<-> ssub * cmtt/m/it}{}
\newcommand{\texfamily}{\fontfamily{cmtex}\selectfont}
\DeclareFontShape{OT1}{cmtt}{bx}{n}
  {<5><6><7><8>cmtt8
   <9>cmbtt9
   <10><10.95><12><14.4><17.28><20.74><24.88>cmbtt10}{}
\DeclareFontShape{OT1}{cmtex}{bx}{n}
  {<-> ssub * cmtt/bx/n}{}
\newcommand{\tex}[1]{\text{\texfamily#1}}	% NEU

\newcommand{\Sp}{\hskip.33334em\relax}


\newcommand{\Conid}[1]{\mathit{#1}}
\newcommand{\Varid}[1]{\mathit{#1}}
\newcommand{\anonymous}{\kern0.06em \vbox{\hrule\@width.5em}}
\newcommand{\plus}{\mathbin{+\!\!\!+}}
\newcommand{\bind}{\mathbin{>\!\!\!>\mkern-6.7mu=}}
\newcommand{\rbind}{\mathbin{=\mkern-6.7mu<\!\!\!<}}% suggested by Neil Mitchell
\newcommand{\sequ}{\mathbin{>\!\!\!>}}
\renewcommand{\leq}{\leqslant}
\renewcommand{\geq}{\geqslant}
\usepackage{polytable}

%mathindent has to be defined
\@ifundefined{mathindent}%
  {\newdimen\mathindent\mathindent\leftmargini}%
  {}%

\def\resethooks{%
  \global\let\SaveRestoreHook\empty
  \global\let\ColumnHook\empty}
\newcommand*{\savecolumns}[1][default]%
  {\g@addto@macro\SaveRestoreHook{\savecolumns[#1]}}
\newcommand*{\restorecolumns}[1][default]%
  {\g@addto@macro\SaveRestoreHook{\restorecolumns[#1]}}
\newcommand*{\aligncolumn}[2]%
  {\g@addto@macro\ColumnHook{\column{#1}{#2}}}

\resethooks

\newcommand{\onelinecommentchars}{\quad-{}- }
\newcommand{\commentbeginchars}{\enskip\{-}
\newcommand{\commentendchars}{-\}\enskip}

\newcommand{\visiblecomments}{%
  \let\onelinecomment=\onelinecommentchars
  \let\commentbegin=\commentbeginchars
  \let\commentend=\commentendchars}

\newcommand{\invisiblecomments}{%
  \let\onelinecomment=\empty
  \let\commentbegin=\empty
  \let\commentend=\empty}

\visiblecomments

\newlength{\blanklineskip}
\setlength{\blanklineskip}{0.66084ex}

\newcommand{\hsindent}[1]{\quad}% default is fixed indentation
\let\hspre\empty
\let\hspost\empty
\newcommand{\NB}{\textbf{NB}}
\newcommand{\Todo}[1]{$\langle$\textbf{To do:}~#1$\rangle$}

\EndFmtInput
\makeatother
%
%
%
%
%
%
% This package provides two environments suitable to take the place
% of hscode, called "plainhscode" and "arrayhscode". 
%
% The plain environment surrounds each code block by vertical space,
% and it uses \abovedisplayskip and \belowdisplayskip to get spacing
% similar to formulas. Note that if these dimensions are changed,
% the spacing around displayed math formulas changes as well.
% All code is indented using \leftskip.
%
% Changed 19.08.2004 to reflect changes in colorcode. Should work with
% CodeGroup.sty.
%
\ReadOnlyOnce{polycode.fmt}%
\makeatletter

\newcommand{\hsnewpar}[1]%
  {{\parskip=0pt\parindent=0pt\par\vskip #1\noindent}}

% can be used, for instance, to redefine the code size, by setting the
% command to \small or something alike
\newcommand{\hscodestyle}{}

% The command \sethscode can be used to switch the code formatting
% behaviour by mapping the hscode environment in the subst directive
% to a new LaTeX environment.

\newcommand{\sethscode}[1]%
  {\expandafter\let\expandafter\hscode\csname #1\endcsname
   \expandafter\let\expandafter\endhscode\csname end#1\endcsname}

% "compatibility" mode restores the non-polycode.fmt layout.

\newenvironment{compathscode}%
  {\par\noindent
   \advance\leftskip\mathindent
   \hscodestyle
   \let\\=\@normalcr
   \let\hspre\(\let\hspost\)%
   \pboxed}%
  {\endpboxed\)%
   \par\noindent
   \ignorespacesafterend}

\newcommand{\compaths}{\sethscode{compathscode}}

% "plain" mode is the proposed default.
% It should now work with \centering.
% This required some changes. The old version
% is still available for reference as oldplainhscode.

\newenvironment{plainhscode}%
  {\hsnewpar\abovedisplayskip
   \advance\leftskip\mathindent
   \hscodestyle
   \let\hspre\(\let\hspost\)%
   \pboxed}%
  {\endpboxed%
   \hsnewpar\belowdisplayskip
   \ignorespacesafterend}

\newenvironment{oldplainhscode}%
  {\hsnewpar\abovedisplayskip
   \advance\leftskip\mathindent
   \hscodestyle
   \let\\=\@normalcr
   \(\pboxed}%
  {\endpboxed\)%
   \hsnewpar\belowdisplayskip
   \ignorespacesafterend}

% Here, we make plainhscode the default environment.

\newcommand{\plainhs}{\sethscode{plainhscode}}
\newcommand{\oldplainhs}{\sethscode{oldplainhscode}}
\plainhs

% The arrayhscode is like plain, but makes use of polytable's
% parray environment which disallows page breaks in code blocks.

\newenvironment{arrayhscode}%
  {\hsnewpar\abovedisplayskip
   \advance\leftskip\mathindent
   \hscodestyle
   \let\\=\@normalcr
   \(\parray}%
  {\endparray\)%
   \hsnewpar\belowdisplayskip
   \ignorespacesafterend}

\newcommand{\arrayhs}{\sethscode{arrayhscode}}

% The mathhscode environment also makes use of polytable's parray 
% environment. It is supposed to be used only inside math mode 
% (I used it to typeset the type rules in my thesis).

\newenvironment{mathhscode}%
  {\parray}{\endparray}

\newcommand{\mathhs}{\sethscode{mathhscode}}

% texths is similar to mathhs, but works in text mode.

\newenvironment{texthscode}%
  {\(\parray}{\endparray\)}

\newcommand{\texths}{\sethscode{texthscode}}

% The framed environment places code in a framed box.

\def\codeframewidth{\arrayrulewidth}
\RequirePackage{calc}

\newenvironment{framedhscode}%
  {\parskip=\abovedisplayskip\par\noindent
   \hscodestyle
   \arrayrulewidth=\codeframewidth
   \tabular{@{}|p{\linewidth-2\arraycolsep-2\arrayrulewidth-2pt}|@{}}%
   \hline\framedhslinecorrect\\{-1.5ex}%
   \let\endoflinesave=\\
   \let\\=\@normalcr
   \(\pboxed}%
  {\endpboxed\)%
   \framedhslinecorrect\endoflinesave{.5ex}\hline
   \endtabular
   \parskip=\belowdisplayskip\par\noindent
   \ignorespacesafterend}

\newcommand{\framedhslinecorrect}[2]%
  {#1[#2]}

\newcommand{\framedhs}{\sethscode{framedhscode}}

% The inlinehscode environment is an experimental environment
% that can be used to typeset displayed code inline.

\newenvironment{inlinehscode}%
  {\(\def\column##1##2{}%
   \let\>\undefined\let\<\undefined\let\\\undefined
   \newcommand\>[1][]{}\newcommand\<[1][]{}\newcommand\\[1][]{}%
   \def\fromto##1##2##3{##3}%
   \def\nextline{}}{\) }%

\newcommand{\inlinehs}{\sethscode{inlinehscode}}

% The joincode environment is a separate environment that
% can be used to surround and thereby connect multiple code
% blocks.

\newenvironment{joincode}%
  {\let\orighscode=\hscode
   \let\origendhscode=\endhscode
   \def\endhscode{\def\hscode{\endgroup\def\@currenvir{hscode}\\}\begingroup}
   %\let\SaveRestoreHook=\empty
   %\let\ColumnHook=\empty
   %\let\resethooks=\empty
   \orighscode\def\hscode{\endgroup\def\@currenvir{hscode}}}%
  {\origendhscode
   \global\let\hscode=\orighscode
   \global\let\endhscode=\origendhscode}%

\makeatother
\EndFmtInput
%
\begin{document}

First, we want some imports:

\section{Parser}

\begin{hscode}\SaveRestoreHook
\column{B}{@{}>{\hspre}l<{\hspost}@{}}%
\column{E}{@{}>{\hspre}l<{\hspost}@{}}%
\>[B]{}\mathbf{module}\;\Conid{Main}\;\mathbf{where}{}\<[E]%
\\
\>[B]{}\mathbf{import}\;\Conid{\Conid{Text}.Megaparsec}\;\Varid{hiding}\;(\Conid{State}){}\<[E]%
\\
\>[B]{}\mathbf{import}\;\Conid{\Conid{Text}.\Conid{Megaparsec}.Char}{}\<[E]%
\\
\>[B]{}\mathbf{import}\;\Varid{qualified}\;\Conid{\Conid{Text}.\Conid{Megaparsec}.\Conid{Char}.Lexer}\;\Varid{as}\;\Conid{L}{}\<[E]%
\\
\>[B]{}\mathbf{import}\;\Varid{qualified}\;\Conid{\Conid{Data}.Text}\;\Varid{as}\;\Conid{T}{}\<[E]%
\\
\>[B]{}\mathbf{import}\;\Varid{qualified}\;\Conid{\Conid{Data}.\Conid{Text}.IO}\;\Varid{as}\;\Conid{T}{}\<[E]%
\\
\>[B]{}\mathbf{import}\;\Conid{\Conid{Data}.Void}{}\<[E]%
\\
\>[B]{}\mathbf{import}\;\Conid{\Conid{Data}.Foldable}{}\<[E]%
\\
\>[B]{}\mathbf{import}\;\Varid{qualified}\;\Conid{\Conid{Data}.Map}\;\Varid{as}\;\Conid{Map}{}\<[E]%
\\
\>[B]{}\mathbf{import}\;\Varid{qualified}\;\Conid{\Conid{Data}.Set}\;\Varid{as}\;\Conid{Set}{}\<[E]%
\\
\>[B]{}\mathbf{import}\;\Conid{\Conid{Control}.\Conid{Monad}.State}{}\<[E]%
\\
\>[B]{}\mathbf{import}\;\Conid{\Conid{Control}.Monad}\;(\Varid{void}){}\<[E]%
\ColumnHook
\end{hscode}\resethooks

Then, we add some boilerplate:

\begin{hscode}\SaveRestoreHook
\column{B}{@{}>{\hspre}l<{\hspost}@{}}%
\column{E}{@{}>{\hspre}l<{\hspost}@{}}%
\>[B]{}\Varid{lexeme}\mathbin{::}\Conid{Parser}\;\Varid{a}\to \Conid{Parser}\;\Varid{a}{}\<[E]%
\\
\>[B]{}\Varid{lexeme}\mathrel{=}\Varid{\Conid{L}.lexeme}\;\Varid{space}{}\<[E]%
\\[\blanklineskip]%
\>[B]{}\Varid{symbol}\mathbin{::}\Conid{\Conid{T}.Text}\to \Conid{Parser}\;\Conid{\Conid{T}.Text}{}\<[E]%
\\
\>[B]{}\Varid{symbol}\mathrel{=}\Varid{\Conid{L}.symbol}\;\Varid{space}{}\<[E]%
\\[\blanklineskip]%
\>[B]{}\Varid{packSymbol}\mathbin{::}\Conid{String}\to \Conid{Parser}\;\Conid{\Conid{T}.Text}{}\<[E]%
\\
\>[B]{}\Varid{packSymbol}\mathrel{=}\Varid{symbol}\mathbin{\circ}\Varid{\Conid{T}.pack}{}\<[E]%
\\[\blanklineskip]%
\>[B]{}\Varid{parens}\mathbin{::}\Conid{Parser}\;\Varid{a}\to \Conid{Parser}\;\Varid{a}{}\<[E]%
\\
\>[B]{}\Varid{parens}\mathrel{=}\Varid{between}\;(\Varid{packSymbol}\;\text{\ttfamily \char34 (\char34})\;(\Varid{packSymbol}\;\text{\ttfamily \char34 )\char34}){}\<[E]%
\ColumnHook
\end{hscode}\resethooks

We rewrite the following grammar into haskell:

\begin{tabbing}\ttfamily
~\char60{}formula\char62{}~\char58{}\char58{}\char61{}~\char96{}\char40{}\char39{}~\char96{}and\char39{}~\char60{}formula\char62{}~\char60{}formula\char62{}~\char96{}\char41{}\char39{}\\
\ttfamily ~~~~~~~~~~~\char124{}~\char96{}\char40{}\char39{}~\char96{}or\char39{}~\char60{}formula\char62{}~\char60{}formula\char62{}~\char96{}\char41{}\char39{}\\
\ttfamily ~~~~~~~~~~~\char124{}~\char96{}\char40{}\char39{}~\char96{}not\char39{}~\char60{}variable\char62{}~\char96{}\char41{}\char39{}~\\
\ttfamily ~~~~~~~~~~~\char124{}~\char60{}variable\char62{}
\end{tabbing}

\begin{hscode}\SaveRestoreHook
\column{B}{@{}>{\hspre}l<{\hspost}@{}}%
\column{12}{@{}>{\hspre}l<{\hspost}@{}}%
\column{13}{@{}>{\hspre}l<{\hspost}@{}}%
\column{16}{@{}>{\hspre}c<{\hspost}@{}}%
\column{16E}{@{}l@{}}%
\column{23}{@{}>{\hspre}l<{\hspost}@{}}%
\column{E}{@{}>{\hspre}l<{\hspost}@{}}%
\>[B]{}\Varid{formula}\mathbin{::}\Conid{Parser}\;\Conid{Formula}{}\<[E]%
\\
\>[B]{}\Varid{formula}\mathrel{=}{}\<[12]%
\>[12]{}\Varid{try}\;(\Varid{parens}\mathbin{\$}\mathbf{do}\;\Varid{packSymbol}\;\text{\ttfamily \char34 and\char34}\sequ \Conid{FAnd}\mathbin{<\$>}\Varid{formula}\mathbin{<*>}\Varid{formula}{}\<[E]%
\\
\>[12]{}\hsindent{11}{}\<[23]%
\>[23]{}\mathbin{<|>}\mathbf{do}\;\Varid{packSymbol}\;\text{\ttfamily \char34 or\char34}\sequ \Conid{FOr}\mathbin{<\$>}\Varid{formula}\mathbin{<*>}\Varid{formula}{}\<[E]%
\\
\>[12]{}\hsindent{11}{}\<[23]%
\>[23]{}\mathbin{<|>}\mathbf{do}\;\Varid{packSymbol}\;\text{\ttfamily \char34 not\char34}\sequ \Conid{FNeg}\mathbin{\circ}\Conid{FVar}\mathbin{<\$>}\Varid{variable}{}\<[E]%
\\
\>[12]{}\hsindent{4}{}\<[16]%
\>[16]{}){}\<[16E]%
\\
\>[12]{}\hsindent{1}{}\<[13]%
\>[13]{}\mathbin{<|>}\mathbf{do}\;\Conid{FVar}\mathbin{<\$>}\Varid{variable}{}\<[E]%
\\[\blanklineskip]%
\>[B]{}\Varid{variable}\mathbin{::}\Conid{Parser}\;\Conid{Variable}{}\<[E]%
\\
\>[B]{}\Varid{variable}\mathrel{=}\Varid{lexeme}\mathbin{\$}(\Varid{pure}\mathbin{<\$>}\Varid{letterChar})\mathbin{<>}\Varid{many}\;\Varid{alphaNumChar}{}\<[E]%
\ColumnHook
\end{hscode}\resethooks

And then we define the supporting types:

\begin{hscode}\SaveRestoreHook
\column{B}{@{}>{\hspre}l<{\hspost}@{}}%
\column{14}{@{}>{\hspre}l<{\hspost}@{}}%
\column{E}{@{}>{\hspre}l<{\hspost}@{}}%
\>[B]{}\mathbf{type}\;\Conid{Parser}\mathrel{=}\Conid{Parsec}\;\Conid{Void}\;\Conid{\Conid{T}.Text}{}\<[E]%
\\[\blanklineskip]%
\>[B]{}\mathbf{data}\;\Conid{Formula}\mathrel{=}\Conid{FAnd}\;\Conid{Formula}\;\Conid{Formula}{}\<[E]%
\\
\>[B]{}\hsindent{14}{}\<[14]%
\>[14]{}\mid \Conid{FOr}\;\Conid{Formula}\;\Conid{Formula}{}\<[E]%
\\
\>[B]{}\hsindent{14}{}\<[14]%
\>[14]{}\mid \Conid{FNeg}\;\Conid{Formula}{}\<[E]%
\\
\>[B]{}\hsindent{14}{}\<[14]%
\>[14]{}\mid \Conid{FVar}\;\Conid{Variable}{}\<[E]%
\\
\>[B]{}\hsindent{14}{}\<[14]%
\>[14]{}\mathbf{deriving}\;(\Conid{Eq},\Conid{Ord},\Conid{Show}){}\<[E]%
\\[\blanklineskip]%
\>[B]{}\mathbf{type}\;\Conid{Variable}\mathrel{=}\Conid{String}{}\<[E]%
\ColumnHook
\end{hscode}\resethooks

\section{Encoding}

Now we can define the encoding:

\begin{hscode}\SaveRestoreHook
\column{B}{@{}>{\hspre}l<{\hspost}@{}}%
\column{5}{@{}>{\hspre}l<{\hspost}@{}}%
\column{9}{@{}>{\hspre}l<{\hspost}@{}}%
\column{13}{@{}>{\hspre}l<{\hspost}@{}}%
\column{E}{@{}>{\hspre}l<{\hspost}@{}}%
\>[B]{}\Varid{encode}\;\Varid{f}\mathrel{=}\mathbf{do}{}\<[E]%
\\
\>[B]{}\hsindent{5}{}\<[5]%
\>[5]{}\Varid{name}\leftarrow \Varid{encode'}\;\Varid{f}{}\<[E]%
\\
\>[B]{}\hsindent{5}{}\<[5]%
\>[5]{}\Varid{modify}\;(\lambda \Varid{state}\to \Varid{state}\;\{\mskip1.5mu \Varid{cnfRepr}\mathrel{=}[\mskip1.5mu \Varid{name}\mskip1.5mu]\mathbin{`\Varid{\Conid{Set}.insert}`}\Varid{cnfRepr}\;\Varid{state}\mskip1.5mu\}){}\<[E]%
\\[\blanklineskip]%
\>[B]{}\Varid{encode'}\mathbin{::}\Conid{Formula}\to \Conid{Encoder}\;\Conid{Int}{}\<[E]%
\\
\>[B]{}\Varid{encode'}\;\Varid{f}\mathord{@}(\Conid{FAnd}\;\Varid{left}\;\Varid{right})\mathrel{=}\mathbf{do}{}\<[E]%
\\
\>[B]{}\hsindent{5}{}\<[5]%
\>[5]{}\Varid{leftName}\leftarrow \Varid{encode'}\;\Varid{left}{}\<[E]%
\\
\>[B]{}\hsindent{5}{}\<[5]%
\>[5]{}\Varid{rightName}\leftarrow \Varid{encode'}\;\Varid{right}{}\<[E]%
\\
\>[B]{}\hsindent{5}{}\<[5]%
\>[5]{}\Varid{name}\leftarrow \Varid{getName}\;\Varid{f}{}\<[E]%
\\
\>[B]{}\hsindent{5}{}\<[5]%
\>[5]{}\Varid{state}\mathord{@}\Conid{EncoderState}\;\{\mskip1.5mu \Varid{cnfRepr}\mathrel{=}\Varid{cnf}\mskip1.5mu\}\leftarrow \Varid{get}{}\<[E]%
\\
\>[B]{}\hsindent{5}{}\<[5]%
\>[5]{}\Varid{put}\;\Varid{state}\;\{\mskip1.5mu \Varid{cnfRepr}\mathrel{=}\Varid{foldr}\;\Varid{\Conid{Set}.insert}\;\Varid{cnf}\;[\mskip1.5mu [\mskip1.5mu \mathbin{-}\Varid{name},\Varid{leftName}\mskip1.5mu],[\mskip1.5mu \mathbin{-}\Varid{name},\Varid{rightName}\mskip1.5mu],[\mskip1.5mu \mathbin{-}\Varid{leftName},\mathbin{-}\Varid{rightName},\Varid{name}\mskip1.5mu]\mskip1.5mu]\mskip1.5mu\}{}\<[E]%
\\
\>[B]{}\hsindent{5}{}\<[5]%
\>[5]{}\Varid{return}\;\Varid{name}{}\<[E]%
\\[\blanklineskip]%
\>[B]{}\Varid{encode'}\;(\Conid{FOr}\;\Varid{left}\;\Varid{right})\mathrel{=}\Varid{encode'}\mathbin{\circ}\Conid{FNeg}\mathbin{\$}\Conid{FAnd}\;(\Conid{FNeg}\;\Varid{left})\;(\Conid{FNeg}\;\Varid{right}){}\<[E]%
\\
\>[B]{}\Varid{encode'}\;(\Conid{FNeg}\;\Varid{f})\mathrel{=}\Varid{negate}\mathbin{<\$>}\Varid{encode'}\;\Varid{f}{}\<[E]%
\\
\>[B]{}\Varid{encode'}\;\Varid{f}\mathord{@}\Conid{FVar}\;\{\mskip1.5mu \mskip1.5mu\}\mathrel{=}\Varid{getName}\;\Varid{f}{}\<[E]%
\\[\blanklineskip]%
\>[B]{}\Varid{getName}\mathbin{::}\Conid{Formula}\to \Conid{Encoder}\;\Conid{Int}{}\<[E]%
\\
\>[B]{}\Varid{getName}\;\Varid{f}\mathrel{=}\mathbf{do}{}\<[E]%
\\
\>[B]{}\hsindent{5}{}\<[5]%
\>[5]{}\Varid{state}\mathord{@}\Conid{EncoderState}\;\{\mskip1.5mu \Varid{formulaNames}\mathrel{=}\Varid{names}\mskip1.5mu\}\leftarrow \Varid{get}{}\<[E]%
\\
\>[B]{}\hsindent{5}{}\<[5]%
\>[5]{}\mathbf{case}\;\Varid{f}\mathbin{`\Varid{\Conid{Map}.lookup}`}\Varid{names}\;\mathbf{of}{}\<[E]%
\\
\>[5]{}\hsindent{4}{}\<[9]%
\>[9]{}\Conid{Just}\;\Varid{i}\to \Varid{return}\;\Varid{i}{}\<[E]%
\\
\>[5]{}\hsindent{4}{}\<[9]%
\>[9]{}\Conid{Nothing}\to \mathbf{do}{}\<[E]%
\\
\>[9]{}\hsindent{4}{}\<[13]%
\>[13]{}\mathbf{let}\;\Varid{name}\mathrel{=}\Varid{\Conid{Map}.size}\;\Varid{names}\mathbin{+}\mathrm{1}{}\<[E]%
\\
\>[9]{}\hsindent{4}{}\<[13]%
\>[13]{}\Varid{put}\;\Varid{state}\;\{\mskip1.5mu \Varid{formulaNames}\mathrel{=}\Varid{\Conid{Map}.insert}\;\Varid{f}\;\Varid{name}\;\Varid{names}\mskip1.5mu\}{}\<[E]%
\\
\>[9]{}\hsindent{4}{}\<[13]%
\>[13]{}\Varid{return}\;\Varid{name}{}\<[E]%
\ColumnHook
\end{hscode}\resethooks

\begin{hscode}\SaveRestoreHook
\column{B}{@{}>{\hspre}l<{\hspost}@{}}%
\column{5}{@{}>{\hspre}l<{\hspost}@{}}%
\column{E}{@{}>{\hspre}l<{\hspost}@{}}%
\>[B]{}\mathbf{type}\;\Conid{Encoder}\mathrel{=}\Conid{State}\;\Conid{EncoderState}{}\<[E]%
\\[\blanklineskip]%
\>[B]{}\mathbf{data}\;\Conid{EncoderState}\mathrel{=}\Conid{EncoderState}{}\<[E]%
\\
\>[B]{}\hsindent{5}{}\<[5]%
\>[5]{}\{\mskip1.5mu \Varid{formulaNames}\mathbin{::}\Conid{\Conid{Map}.Map}\;\Conid{Formula}\;\Conid{Int}{}\<[E]%
\\
\>[B]{}\hsindent{5}{}\<[5]%
\>[5]{},\Varid{cnfRepr}\mathbin{::}\Conid{\Conid{Set}.Set}\;[\mskip1.5mu \Conid{Int}\mskip1.5mu]{}\<[E]%
\\
\>[B]{}\hsindent{5}{}\<[5]%
\>[5]{}\mskip1.5mu\}{}\<[E]%
\ColumnHook
\end{hscode}\resethooks


\begin{hscode}\SaveRestoreHook
\column{B}{@{}>{\hspre}l<{\hspost}@{}}%
\column{5}{@{}>{\hspre}l<{\hspost}@{}}%
\column{9}{@{}>{\hspre}l<{\hspost}@{}}%
\column{13}{@{}>{\hspre}l<{\hspost}@{}}%
\column{E}{@{}>{\hspre}l<{\hspost}@{}}%
\>[B]{}\Varid{prettyPrint}\mathbin{::}\Conid{EncoderState}\to \Conid{IO}\;(){}\<[E]%
\\
\>[B]{}\Varid{prettyPrint}\;\Conid{EncoderState}\;\{\mskip1.5mu \Varid{formulaNames}\mathrel{=}\Varid{names},\Varid{cnfRepr}\mathrel{=}\Varid{cnf}\mskip1.5mu\}\mathrel{=}\mathbf{do}{}\<[E]%
\\
\>[B]{}\hsindent{5}{}\<[5]%
\>[5]{}\Varid{originalCount}\leftarrow \Varid{sequence}{}\<[E]%
\\
\>[5]{}\hsindent{4}{}\<[9]%
\>[9]{}[\mskip1.5mu \mathbf{case}\;\Varid{key}\;\mathbf{of}{}\<[E]%
\\
\>[9]{}\hsindent{4}{}\<[13]%
\>[13]{}\Conid{FVar}\;\Varid{name}\to \Varid{putStrLn}\;(\text{\ttfamily \char34 c~\char34}\plus \Varid{name}\plus \text{\ttfamily \char34 ~=~\char34}\plus \Varid{show}\;\Varid{value})\sequ \Varid{return}\;\mathrm{1}{}\<[E]%
\\
\>[9]{}\hsindent{4}{}\<[13]%
\>[13]{}\anonymous \to \Varid{return}\;\mathrm{0}{}\<[E]%
\\
\>[5]{}\hsindent{4}{}\<[9]%
\>[9]{}\mid (\Varid{key},\Varid{value})\leftarrow \Varid{\Conid{Map}.toList}\;\Varid{names}{}\<[E]%
\\
\>[5]{}\hsindent{4}{}\<[9]%
\>[9]{}\mskip1.5mu]{}\<[E]%
\\
\>[B]{}\hsindent{5}{}\<[5]%
\>[5]{}\mathbf{let}\;\Varid{totalCount}\mathrel{=}\Varid{\Conid{Map}.size}\;\Varid{names}{}\<[E]%
\\
\>[B]{}\hsindent{5}{}\<[5]%
\>[5]{}\Varid{putStrLn}\mathbin{\$}\text{\ttfamily \char34 c~\$root~=~\char34}\plus \Varid{show}\;\Varid{totalCount}{}\<[E]%
\\
\>[B]{}\hsindent{5}{}\<[5]%
\>[5]{}\Varid{putStrLn}\mathbin{\$}\text{\ttfamily \char34 p~cnf~\char34}\plus \Varid{show}\;\Varid{totalCount}\plus \text{\ttfamily '~'}\mathbin{:}\Varid{show}\;(\Varid{length}\;\Varid{cnf}){}\<[E]%
\\
\>[B]{}\hsindent{5}{}\<[5]%
\>[5]{}\Varid{sequence\char95 }\;{}\<[E]%
\\
\>[5]{}\hsindent{4}{}\<[9]%
\>[9]{}[\mskip1.5mu \Varid{sequence\char95 }\;[\mskip1.5mu \Varid{putStr}\;(\Varid{show}\;\Varid{item}\plus \text{\ttfamily \char34 ~\char34})\mid \Varid{item}\leftarrow \Varid{clause}\mskip1.5mu]\sequ \Varid{putStrLn}\;\text{\ttfamily \char34 0\char34}{}\<[E]%
\\
\>[5]{}\hsindent{4}{}\<[9]%
\>[9]{}\mid \Varid{clause}\leftarrow \Varid{toList}\;\Varid{cnf}{}\<[E]%
\\
\>[5]{}\hsindent{4}{}\<[9]%
\>[9]{}\mskip1.5mu]{}\<[E]%
\\[\blanklineskip]%
\>[B]{}\Varid{main}\mathbin{::}\Conid{IO}\;(){}\<[E]%
\\
\>[B]{}\Varid{main}\mathrel{=}\mathbf{do}{}\<[E]%
\\
\>[B]{}\hsindent{5}{}\<[5]%
\>[5]{}\Varid{contents}\leftarrow \Varid{\Conid{T}.getContents}{}\<[E]%
\\
\>[B]{}\hsindent{5}{}\<[5]%
\>[5]{}\mathbf{case}\;\Varid{parse}\;\Varid{formula}\;\text{\ttfamily \char34 vstup\char34}\;\Varid{contents}\;\mathbf{of}{}\<[E]%
\\
\>[5]{}\hsindent{4}{}\<[9]%
\>[9]{}\Conid{Right}\;\Varid{f}\to \mathbf{do}{}\<[E]%
\\
\>[9]{}\hsindent{4}{}\<[13]%
\>[13]{}\mathbf{let}\;\Varid{result}\mathrel{=}\Varid{execState}\;(\Varid{encode}\;\Varid{f})\;\Conid{EncoderState}\;\{\mskip1.5mu \Varid{formulaNames}\mathrel{=}\Varid{mempty},\Varid{cnfRepr}\mathrel{=}\Varid{mempty}\mskip1.5mu\}{}\<[E]%
\\
\>[9]{}\hsindent{4}{}\<[13]%
\>[13]{}\Varid{prettyPrint}\;\Varid{result}{}\<[E]%
\\
\>[5]{}\hsindent{4}{}\<[9]%
\>[9]{}\Conid{Left}\;\Varid{e}\to \Varid{print}\;\Varid{e}{}\<[E]%
\ColumnHook
\end{hscode}\resethooks


\end{document}
