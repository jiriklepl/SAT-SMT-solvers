
\documentclass{article}
%% ODER: format ==         = "\mathrel{==}"
%% ODER: format /=         = "\neq "
%
%
\makeatletter
\@ifundefined{lhs2tex.lhs2tex.sty.read}%
  {\@namedef{lhs2tex.lhs2tex.sty.read}{}%
   \newcommand\SkipToFmtEnd{}%
   \newcommand\EndFmtInput{}%
   \long\def\SkipToFmtEnd#1\EndFmtInput{}%
  }\SkipToFmtEnd

\newcommand\ReadOnlyOnce[1]{\@ifundefined{#1}{\@namedef{#1}{}}\SkipToFmtEnd}
\usepackage{amstext}
\usepackage{amssymb}
\usepackage{stmaryrd}
\DeclareFontFamily{OT1}{cmtex}{}
\DeclareFontShape{OT1}{cmtex}{m}{n}
  {<5><6><7><8>cmtex8
   <9>cmtex9
   <10><10.95><12><14.4><17.28><20.74><24.88>cmtex10}{}
\DeclareFontShape{OT1}{cmtex}{m}{it}
  {<-> ssub * cmtt/m/it}{}
\newcommand{\texfamily}{\fontfamily{cmtex}\selectfont}
\DeclareFontShape{OT1}{cmtt}{bx}{n}
  {<5><6><7><8>cmtt8
   <9>cmbtt9
   <10><10.95><12><14.4><17.28><20.74><24.88>cmbtt10}{}
\DeclareFontShape{OT1}{cmtex}{bx}{n}
  {<-> ssub * cmtt/bx/n}{}
\newcommand{\tex}[1]{\text{\texfamily#1}}	% NEU

\newcommand{\Sp}{\hskip.33334em\relax}


\newcommand{\Conid}[1]{\mathit{#1}}
\newcommand{\Varid}[1]{\mathit{#1}}
\newcommand{\anonymous}{\kern0.06em \vbox{\hrule\@width.5em}}
\newcommand{\plus}{\mathbin{+\!\!\!+}}
\newcommand{\bind}{\mathbin{>\!\!\!>\mkern-6.7mu=}}
\newcommand{\rbind}{\mathbin{=\mkern-6.7mu<\!\!\!<}}% suggested by Neil Mitchell
\newcommand{\sequ}{\mathbin{>\!\!\!>}}
\renewcommand{\leq}{\leqslant}
\renewcommand{\geq}{\geqslant}
\usepackage{polytable}

%mathindent has to be defined
\@ifundefined{mathindent}%
  {\newdimen\mathindent\mathindent\leftmargini}%
  {}%

\def\resethooks{%
  \global\let\SaveRestoreHook\empty
  \global\let\ColumnHook\empty}
\newcommand*{\savecolumns}[1][default]%
  {\g@addto@macro\SaveRestoreHook{\savecolumns[#1]}}
\newcommand*{\restorecolumns}[1][default]%
  {\g@addto@macro\SaveRestoreHook{\restorecolumns[#1]}}
\newcommand*{\aligncolumn}[2]%
  {\g@addto@macro\ColumnHook{\column{#1}{#2}}}

\resethooks

\newcommand{\onelinecommentchars}{\quad-{}- }
\newcommand{\commentbeginchars}{\enskip\{-}
\newcommand{\commentendchars}{-\}\enskip}

\newcommand{\visiblecomments}{%
  \let\onelinecomment=\onelinecommentchars
  \let\commentbegin=\commentbeginchars
  \let\commentend=\commentendchars}

\newcommand{\invisiblecomments}{%
  \let\onelinecomment=\empty
  \let\commentbegin=\empty
  \let\commentend=\empty}

\visiblecomments

\newlength{\blanklineskip}
\setlength{\blanklineskip}{0.66084ex}

\newcommand{\hsindent}[1]{\quad}% default is fixed indentation
\let\hspre\empty
\let\hspost\empty
\newcommand{\NB}{\textbf{NB}}
\newcommand{\Todo}[1]{$\langle$\textbf{To do:}~#1$\rangle$}

\EndFmtInput
\makeatother
%
%
%
%
%
%
% This package provides two environments suitable to take the place
% of hscode, called "plainhscode" and "arrayhscode". 
%
% The plain environment surrounds each code block by vertical space,
% and it uses \abovedisplayskip and \belowdisplayskip to get spacing
% similar to formulas. Note that if these dimensions are changed,
% the spacing around displayed math formulas changes as well.
% All code is indented using \leftskip.
%
% Changed 19.08.2004 to reflect changes in colorcode. Should work with
% CodeGroup.sty.
%
\ReadOnlyOnce{polycode.fmt}%
\makeatletter

\newcommand{\hsnewpar}[1]%
  {{\parskip=0pt\parindent=0pt\par\vskip #1\noindent}}

% can be used, for instance, to redefine the code size, by setting the
% command to \small or something alike
\newcommand{\hscodestyle}{}

% The command \sethscode can be used to switch the code formatting
% behaviour by mapping the hscode environment in the subst directive
% to a new LaTeX environment.

\newcommand{\sethscode}[1]%
  {\expandafter\let\expandafter\hscode\csname #1\endcsname
   \expandafter\let\expandafter\endhscode\csname end#1\endcsname}

% "compatibility" mode restores the non-polycode.fmt layout.

\newenvironment{compathscode}%
  {\par\noindent
   \advance\leftskip\mathindent
   \hscodestyle
   \let\\=\@normalcr
   \let\hspre\(\let\hspost\)%
   \pboxed}%
  {\endpboxed\)%
   \par\noindent
   \ignorespacesafterend}

\newcommand{\compaths}{\sethscode{compathscode}}

% "plain" mode is the proposed default.
% It should now work with \centering.
% This required some changes. The old version
% is still available for reference as oldplainhscode.

\newenvironment{plainhscode}%
  {\hsnewpar\abovedisplayskip
   \advance\leftskip\mathindent
   \hscodestyle
   \let\hspre\(\let\hspost\)%
   \pboxed}%
  {\endpboxed%
   \hsnewpar\belowdisplayskip
   \ignorespacesafterend}

\newenvironment{oldplainhscode}%
  {\hsnewpar\abovedisplayskip
   \advance\leftskip\mathindent
   \hscodestyle
   \let\\=\@normalcr
   \(\pboxed}%
  {\endpboxed\)%
   \hsnewpar\belowdisplayskip
   \ignorespacesafterend}

% Here, we make plainhscode the default environment.

\newcommand{\plainhs}{\sethscode{plainhscode}}
\newcommand{\oldplainhs}{\sethscode{oldplainhscode}}
\plainhs

% The arrayhscode is like plain, but makes use of polytable's
% parray environment which disallows page breaks in code blocks.

\newenvironment{arrayhscode}%
  {\hsnewpar\abovedisplayskip
   \advance\leftskip\mathindent
   \hscodestyle
   \let\\=\@normalcr
   \(\parray}%
  {\endparray\)%
   \hsnewpar\belowdisplayskip
   \ignorespacesafterend}

\newcommand{\arrayhs}{\sethscode{arrayhscode}}

% The mathhscode environment also makes use of polytable's parray 
% environment. It is supposed to be used only inside math mode 
% (I used it to typeset the type rules in my thesis).

\newenvironment{mathhscode}%
  {\parray}{\endparray}

\newcommand{\mathhs}{\sethscode{mathhscode}}

% texths is similar to mathhs, but works in text mode.

\newenvironment{texthscode}%
  {\(\parray}{\endparray\)}

\newcommand{\texths}{\sethscode{texthscode}}

% The framed environment places code in a framed box.

\def\codeframewidth{\arrayrulewidth}
\RequirePackage{calc}

\newenvironment{framedhscode}%
  {\parskip=\abovedisplayskip\par\noindent
   \hscodestyle
   \arrayrulewidth=\codeframewidth
   \tabular{@{}|p{\linewidth-2\arraycolsep-2\arrayrulewidth-2pt}|@{}}%
   \hline\framedhslinecorrect\\{-1.5ex}%
   \let\endoflinesave=\\
   \let\\=\@normalcr
   \(\pboxed}%
  {\endpboxed\)%
   \framedhslinecorrect\endoflinesave{.5ex}\hline
   \endtabular
   \parskip=\belowdisplayskip\par\noindent
   \ignorespacesafterend}

\newcommand{\framedhslinecorrect}[2]%
  {#1[#2]}

\newcommand{\framedhs}{\sethscode{framedhscode}}

% The inlinehscode environment is an experimental environment
% that can be used to typeset displayed code inline.

\newenvironment{inlinehscode}%
  {\(\def\column##1##2{}%
   \let\>\undefined\let\<\undefined\let\\\undefined
   \newcommand\>[1][]{}\newcommand\<[1][]{}\newcommand\\[1][]{}%
   \def\fromto##1##2##3{##3}%
   \def\nextline{}}{\) }%

\newcommand{\inlinehs}{\sethscode{inlinehscode}}

% The joincode environment is a separate environment that
% can be used to surround and thereby connect multiple code
% blocks.

\newenvironment{joincode}%
  {\let\orighscode=\hscode
   \let\origendhscode=\endhscode
   \def\endhscode{\def\hscode{\endgroup\def\@currenvir{hscode}\\}\begingroup}
   %\let\SaveRestoreHook=\empty
   %\let\ColumnHook=\empty
   %\let\resethooks=\empty
   \orighscode\def\hscode{\endgroup\def\@currenvir{hscode}}}%
  {\origendhscode
   \global\let\hscode=\orighscode
   \global\let\endhscode=\origendhscode}%

\makeatother
\EndFmtInput
%
\begin{document}

First, we want some imports:

\begin{hscode}\SaveRestoreHook
\column{B}{@{}>{\hspre}l<{\hspost}@{}}%
\column{E}{@{}>{\hspre}l<{\hspost}@{}}%
\>[B]{}\mathbf{module}\;\Conid{Main}\;\mathbf{where}{}\<[E]%
\\
\>[B]{}\mathbf{import}\;\Conid{\Conid{Text}.Megaparsec}{}\<[E]%
\\
\>[B]{}\mathbf{import}\;\Conid{\Conid{Text}.\Conid{Megaparsec}.Char}{}\<[E]%
\\
\>[B]{}\mathbf{import}\;\Varid{qualified}\;\Conid{\Conid{Text}.\Conid{Megaparsec}.\Conid{Char}.Lexer}\;\Varid{as}\;\Conid{L}{}\<[E]%
\\
\>[B]{}\mathbf{import}\;\Conid{\Conid{Data}.Text}{}\<[E]%
\\
\>[B]{}\mathbf{import}\;\Conid{\Conid{Data}.Void}{}\<[E]%
\\
\>[B]{}\mathbf{import}\;\Conid{\Conid{Control}.Monad}\;(\Varid{void}){}\<[E]%
\ColumnHook
\end{hscode}\resethooks

Then, we add some boilerplate:

\begin{hscode}\SaveRestoreHook
\column{B}{@{}>{\hspre}l<{\hspost}@{}}%
\column{E}{@{}>{\hspre}l<{\hspost}@{}}%
\>[B]{}\Varid{sc}\mathbin{::}\Conid{Parser}\;(){}\<[E]%
\\
\>[B]{}\Varid{sc}\mathrel{=}\Varid{void}\mathbin{\$}\Varid{many}\;\Varid{spaceChar}{}\<[E]%
\\[\blanklineskip]%
\>[B]{}\Varid{lexeme}\mathbin{::}\Conid{Parser}\;\Varid{a}\to \Conid{Parser}\;\Varid{a}{}\<[E]%
\\
\>[B]{}\Varid{lexeme}\mathrel{=}\Varid{\Conid{L}.lexeme}\;\Varid{sc}{}\<[E]%
\\[\blanklineskip]%
\>[B]{}\Varid{symbol}\mathbin{::}\Conid{Text}\to \Conid{Parser}\;\Conid{Text}{}\<[E]%
\\
\>[B]{}\Varid{symbol}\mathrel{=}\Varid{\Conid{L}.symbol}\;\Varid{sc}{}\<[E]%
\\[\blanklineskip]%
\>[B]{}\Varid{packSymbol}\mathbin{::}\Conid{String}\to \Conid{Parser}\;\Conid{Text}{}\<[E]%
\\
\>[B]{}\Varid{packSymbol}\mathrel{=}\Varid{symbol}\mathbin{\circ}\Varid{pack}{}\<[E]%
\\[\blanklineskip]%
\>[B]{}\Varid{parens}\mathbin{::}\Conid{Parser}\;\Varid{a}\to \Conid{Parser}\;\Varid{a}{}\<[E]%
\\
\>[B]{}\Varid{parens}\mathrel{=}\Varid{between}\;(\Varid{packSymbol}\;\text{\ttfamily \char34 (\char34})\;(\Varid{packSymbol}\;\text{\ttfamily \char34 )\char34}){}\<[E]%
\ColumnHook
\end{hscode}\resethooks

The input file contains a description of a single formula in NNF using a very simplified SMT-LIB format following grammatical rules:

\begin{tabbing}\ttfamily
~\char60{}formula\char62{}~\char58{}\char58{}\char61{}~\char96{}\char40{}\char39{}~\char96{}and\char39{}~\char60{}formula\char62{}~\char60{}formula\char62{}~\char96{}\char41{}\char39{}\\
\ttfamily ~~~~~~~~~~~\char124{}~\char96{}\char40{}\char39{}~\char96{}or\char39{}~\char60{}formula\char62{}~\char60{}formula\char62{}~\char96{}\char41{}\char39{}\\
\ttfamily ~~~~~~~~~~~\char124{}~\char96{}\char40{}\char39{}~\char96{}not\char39{}~\char60{}variable\char62{}~\char96{}\char41{}\char39{}~\\
\ttfamily ~~~~~~~~~~~\char124{}~\char60{}variable\char62{}
\end{tabbing}

We rewrite those into haskell:

\begin{hscode}\SaveRestoreHook
\column{B}{@{}>{\hspre}l<{\hspost}@{}}%
\column{20}{@{}>{\hspre}l<{\hspost}@{}}%
\column{E}{@{}>{\hspre}l<{\hspost}@{}}%
\>[B]{}\Varid{formula}\mathbin{::}\Conid{Parser}\;\Conid{Formula}{}\<[E]%
\\
\>[B]{}\Varid{formula}\mathrel{=}\Varid{parens}\mathbin{\$}\mathbf{do}\;\Varid{packSymbol}\;\text{\ttfamily \char34 and\char34}\sequ \Conid{AndFormula}\mathbin{<\$>}\Varid{formula}\mathbin{<*>}\Varid{formula}{}\<[E]%
\\
\>[B]{}\hsindent{20}{}\<[20]%
\>[20]{}\mathbin{<|>}\mathbf{do}\;\Varid{packSymbol}\;\text{\ttfamily \char34 or\char34}\sequ \Conid{OrFormula}\mathbin{<\$>}\Varid{formula}\mathbin{<*>}\Varid{formula}{}\<[E]%
\\
\>[B]{}\hsindent{20}{}\<[20]%
\>[20]{}\mathbin{<|>}\mathbf{do}\;\Varid{packSymbol}\;\text{\ttfamily \char34 not\char34}\sequ \Conid{NotFormula}\mathbin{<\$>}\Varid{variable}{}\<[E]%
\\[\blanklineskip]%
\>[B]{}\Varid{variable}\mathbin{::}\Conid{Parser}\;\Conid{Variable}{}\<[E]%
\\
\>[B]{}\Varid{variable}\mathrel{=}\Varid{lexeme}\mathbin{\$}(\Varid{pure}\mathbin{<\$>}\Varid{letterChar})\mathbin{<>}\Varid{many}\;\Varid{alphaNumChar}{}\<[E]%
\ColumnHook
\end{hscode}\resethooks

And then we define the supporting types:

\begin{hscode}\SaveRestoreHook
\column{B}{@{}>{\hspre}l<{\hspost}@{}}%
\column{14}{@{}>{\hspre}l<{\hspost}@{}}%
\column{E}{@{}>{\hspre}l<{\hspost}@{}}%
\>[B]{}\mathbf{type}\;\Conid{Parser}\mathrel{=}\Conid{Parsec}\;\Conid{Void}\;\Conid{Text}{}\<[E]%
\\[\blanklineskip]%
\>[B]{}\mathbf{data}\;\Conid{Formula}\mathrel{=}\Conid{AndFormula}\;\Conid{Formula}\;\Conid{Formula}{}\<[E]%
\\
\>[B]{}\hsindent{14}{}\<[14]%
\>[14]{}\mid \Conid{OrFormula}\;\Conid{Formula}\;\Conid{Formula}{}\<[E]%
\\
\>[B]{}\hsindent{14}{}\<[14]%
\>[14]{}\mid \Conid{NotFormula}\;\Conid{Variable}{}\<[E]%
\\[\blanklineskip]%
\>[B]{}\mathbf{type}\;\Conid{Variable}\mathrel{=}\Conid{String}{}\<[E]%
\ColumnHook
\end{hscode}\resethooks

\end{document}
