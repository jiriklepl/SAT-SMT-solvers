\documentclass[a4paper,12pt]{article} % This defines the style of your paper

\usepackage[top = 2.5cm, bottom = 2.5cm, left = 2.5cm, right = 2.5cm]{geometry}

\usepackage[T1]{fontenc}
\usepackage[utf8]{inputenc}
\usepackage[english]{babel}

\usepackage{multirow}
\usepackage{booktabs}

\usepackage{amsmath}
\usepackage{graphicx}

\usepackage{algorithm}
\usepackage{algpseudocode}

\usepackage{listings}
\lstset{language=C}

\usepackage{enumitem}

\newtheorem{definition}{Definition}
\newtheorem{observation}{Observation}[definition]

\usepackage{setspace}
\setlength{\parindent}{0in}
\setlength{\parskip}{.5em}

\begin{document}

\begin{center}
    {\Large \bf Homework 10}
    \vspace{2mm}

    {\bf Jiří Klepl}

\end{center}

\vspace{0.4cm}

\begin{enumerate}[label=\alph*)]
    \item \label{a} The modified program:
    \begin{lstlisting}
char mybuf[256];

void concatenate(
    char buf1[], unsigned len1,
    char buf2[], unsigned len2) {
    // I think there was a bug (>) in the assignment
    if (len1 + len2 >= 256) return; // len1 + len2 < 256

    {   unsigned i = 0;
        if (i != len1) {
            assert(i < len1); // and we don't need (i >= 0)
            i = *;
            assume(i < len1);

            asssert(i < 256); // subsumed by (i < len1)
            asssert(i < len1); // assumed
            mybuf[i] = buf1[i];
            i++;
            assert(i != len1 -> i < len1);
    }   }

    {   unsigned i = 0;
        if (i != len2) {
            assert(i < len2);
            i = *;
            assume(i < len2);

            assert(i + len1 <= 256); // subsumed by:
                // (i < len2 && len1 + len2 < 256)
            assert(i < len2); // assumed
            mybuf[len1 + i] = buf2[i];
            i++;
            assert(i != len2 -> i < len2);
    }   }
}
    \end{lstlisting}
    \item see \ref{a}
    \item They do not hold. Under bit-vector arithmetics, it is possible that $len1 + len2 < 256 \land \exists i: 0 \leq i < len2 \to len1 + i \geq 256$.
\end{enumerate}



\end{document}
