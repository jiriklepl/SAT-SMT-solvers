\documentclass[a4paper,12pt]{article} % This defines the style of your paper

\usepackage[top = 2.5cm, bottom = 2.5cm, left = 2.5cm, right = 2.5cm]{geometry} 

\usepackage[T1]{fontenc}
\usepackage[utf8]{inputenc}
\usepackage[english]{babel}

\usepackage{multirow} % Multirow is for tables with multiple rows within one cell.
\usepackage{booktabs} % For even nicer tables.

\usepackage{amsmath} 
\usepackage{graphicx}

\usepackage{algorithm}
\usepackage{algpseudocode}


\newtheorem{definition}{Definition}
\newtheorem{observation}{Observation}[definition]

\usepackage{setspace}
\setlength{\parindent}{0in}
\setlength{\parskip}{.5em}

\begin{document}

\begin{center}
    {\Large \bf Homework 5}
    \vspace{2mm}

    {\bf Jiří Klepl}

\end{center}

\vspace{0.4cm}

The algorithm will be defined as (stopping at the bottom case):

$B^f \bigodot B^g = (1 - x)(B^f|_{x = 0} \bigodot B^g|_{x = 0}) + x(B^f|_{x = 1} \bigodot B^g|_{x = 1})\\
\text{where } x = min(var(B^f), var(B^g))$


For the demonstration, we will assume that $\bigodot$ takes the lowest priority (sticks the least)

$f \bigodot g: \\
(1 - x_1)(-x_2 \bigodot (\text{if } x_2 \text{ then } 4x_1 \text{ else } x_3+1)) + x_1(2x_2 + 1 \bigodot (\text{if } x_2 \text{ then } 4x_1 \text{ else } x_3+1)) = \\
(1 - x_1)((1 - x_2) (-0 \bigodot x_3 + 1) + x_2(-1 \bigodot 0)) + x_1((1-x_2) (1 \bigodot x_3+1) + x_2(3 \bigodot 4) = \\
(1 - x_1)((1 - x_2) ((1-x_3) (0 \bigodot 1) + x_3 (0 \bigodot 2)) + x_2(-1 \bigodot 0)) + x_1((1-x_2) ((1 - x_3) (1 \bigodot 1) + x_3 (1 \bigodot 2)) + x_2(3 \bigodot 4))$

Written as MTBDD:

\includegraphics{"graph.pdf"}

\end{document}