\documentclass[a4paper,12pt]{article} % This defines the style of your paper

\usepackage[top = 2.5cm, bottom = 2.5cm, left = 2.5cm, right = 2.5cm]{geometry} 

\usepackage[T1]{fontenc}
\usepackage[utf8]{inputenc}
\usepackage[english]{babel}

\usepackage{multirow} % Multirow is for tables with multiple rows within one cell.
\usepackage{booktabs} % For even nicer tables.

\usepackage{amsmath} 
\usepackage{graphicx}

\usepackage{algorithm}
\usepackage{algpseudocode}


\newtheorem{definition}{Definition}
\newtheorem{observation}{Observation}[definition]

\usepackage{setspace}
\setlength{\parindent}{0in}
\setlength{\parskip}{.5em}

\begin{document}

\begin{center}
    {\Large \bf Homework 4}
    \vspace{2mm}

    {\bf Jiří Klepl}

\end{center}

\vspace{0.4cm}

\begin{quote}
    (10 points) Show that every CNF $\varphi$ can be translated in polynomial time into an equisatisfiable
    CNF $\psi$ which is a conjunction of a 2-CNF and a Horn formula. How many additional variables
    are needed?
\end{quote}

Let $T$ denote the translation from $\varphi$ to $\psi$. First, we do few observations:

\begin{itemize}
    \item
    Any variable in $\varphi$ with no positive or with no negative literal occurrence trivially satisfies the clauses it appears in. The same applies to all clauses which contain a positive and a negative literal occurrence of a certain variable.
    \item
    Any clause $C$ in $\varphi$ with less than two positive literals is already a Horn clause and we can simplify the translation from $T(\varphi) = \psi$ to $T(\varphi') \wedge C = \psi$, where $\varphi'$ is $\varphi$ without the clause $C$.
\end{itemize}

So, without loss of generality, we can assume $\varphi$ consists of clauses such that each has at least two positive literals, and that every variable has a positive and a negative occurrence and never in the same clause.

Then we will describe the algorithm for $T(\varphi)$ as follows:

\begin{algorithmic}[1]
    \Function{$T$}{$\varphi$}
    \State $\varphi' \gets \top$
    \While{$\varphi \not\in \textbf{Horn}$}
        \State $x \gets \text{a variable with most positive occurrences in non-Horn clauses}$
        \State $l \gets \text{Lit } x$
        \State $x' \gets \text{fresh variable}$
        \State $\varphi' \gets \varphi' \wedge (x \vee x') \wedge (\neg x \vee \neg x')$
        \State $\varphi \gets \varphi[l := \neg x']$ \Comment rewrite positive occurrences of $x$ with $\neg x'$
    \EndWhile
    \State \Return $\varphi \wedge \varphi'$
    \EndFunction
\end{algorithmic}

The needed number of fresh variables is $\leq|\mathbf{x}|$ (if we consider only the variables with positive occurrences in clauses with at least two positive literals). We could achieve a slightly higher number with tseitin transformation of the maximal positive subsets of clauses.

\end{document}